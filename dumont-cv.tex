%%%%%%%%%%%%%%%%%
% This is an example CV created using altacv.cls (v1.1.5, 1 December 2018) written by
% LianTze Lim (liantze@gmail.com), based on the
% Cv created by BusinessInsider at http://www.businessinsider.my/a-sample-resume-for-marissa-mayer-2016-7/?r=US&IR=T
%
%% It may be distributed and/or modified under the
%% conditions of the LaTeX Project Public License, either version 1.3
%% of this license or (at your option) any later version.
%% The latest version of this license is in
%%    http://www.latex-project.org/lppl.txt
%% and version 1.3 or later is part of all distributions of LaTeX
%% version 2003/12/01 or later.
%%%%%%%%%%%%%%%%

%% If you are using \orcid or academicons
%% icons, make sure you have the academicons
%% option here, and compile with XeLaTeX
%% or LuaLaTeX.
% \documentclass[10pt,a4paper,academicons]{altacv}

%% Use the "normalphoto" option if you want a normal photo instead of cropped to a circle
% \documentclass[10pt,a4paper,normalphoto]{altacv}

\documentclass[10pt,a4paper,ragged2e]{altacv}

%% AltaCV uses the fontawesome and academicon fonts
%% and packages.
%% See texdoc.net/pkg/fontawecome and http://texdoc.net/pkg/academicons for full list of symbols. You MUST compile with XeLaTeX or LuaLaTeX if you want to use academicons.

% Change the page layout if you need to
\geometry{left=2cm,right=10cm,marginparwidth=6.8cm,marginparsep=1.2cm,top=1.25cm,bottom=1.25cm}

% Change the font if you want to, depending on whether
% you're using pdflatex or xelatex/lualatex
\ifxetexorluatex
  % If using xelatex or lualatex:
  \setmainfont{Carlito}
\else
  % If using pdflatex:
  \usepackage[utf8]{inputenc}
  \usepackage[T1]{fontenc}
  \usepackage[default]{lato}
\fi

% Change the colours if you want to
\definecolor{VividPurple}{HTML}{000000}
\definecolor{SlateGrey}{HTML}{2E2E2E}
\definecolor{LightGrey}{HTML}{2E2E2E}
\colorlet{heading}{VividPurple}
\colorlet{accent}{VividPurple}
\colorlet{emphasis}{SlateGrey}
\colorlet{body}{LightGrey}

% Change the bullets for itemize and rating marker
% for \cvskill if you want to
\renewcommand{\itemmarker}{{\small\textbullet}}
\renewcommand{\ratingmarker}{\faCircle}

%% sample.bib contains your publications
\addbibresource{sample.bib}

\begin{document}
\name{Nicole Dumont}
\tagline{Graduate Student}
% Cropped to square from https://en.wikipedia.org/wiki/Marissa_Mayer#/media/File:Marissa_Mayer_May_2014_(cropped).jpg, CC-BY 2.0
%\photo{3.3cm}{profile.jpg}
\personalinfo{%
  % Not all of these are required!
  % You can add your own with \printinfo{symbol}{detail}
  \email{ns2dumont@uwaterloo.ca}
%   \phone{000-00-0000}
%  \mailaddress{Address, Street, 00000 County}
  \location{Waterloo, Canada}
%  \homepage{marissamayr.tumblr.com/}
%  \twitter{@marissamayer}
  \linkedin{linkedin.com/in/nicole-dumont}
   \github{github.com/nsdumont} % I'm just making this up though.
%   \orcid{orcid.org/0000-0000-0000-0000} % Obviously making this up too. If you want to use this field (and also other academicons symbols), add "academicons" option to \documentclass{altacv}
}

%% Make the header extend all the way to the right, if you want.
\begin{fullwidth}
\makecvheader
\end{fullwidth}

%% Depending on your tastes, you may want to make fonts of itemize environments slightly smaller
\AtBeginEnvironment{itemize}{\small}

%% Provide the file name containing the sidebar contents as an optional parameter to \cvsection.
%% You can always just use \marginpar{...} if you do
%% not need to align the top of the contents to any
%% \cvsection title in the "main" bar.

\cvsection[page1sidebar]{Education}
%\cvsection{Education}

\cvevent{Computer Science (PhD)}{University of Waterloo}{Sept 2019 -- Present}{Waterloo, Canada}
\begin{itemize}
\item Studying computational neuroscience, focusing on modelling grid cells and spatial navigation  
\end{itemize}

\divider

\cvevent{Computational Mathematics (Masters of Mathematics)}{University of Waterloo}{Sept 2017 -- April 2019}{Waterloo, Canada}
\begin{itemize}
\item Cumulative average of 90.57 \% 
\item Completed a masters research paper on robust optimization of an asset pricing model used to price carbon emissions.
\item Courses on optimization, computational statistics, numerical analysis, PDEs, and computational neuroscience.
\end{itemize}

\divider

\cvevent{Honors Mathematics and Physics (Bachelors of Science) }{McMaster University}{Sept 2012 -- April 2017}{Hamilton, Canada}
\begin{itemize}
\item Cumulative average of 10.5/12.0 (GPA 3.8)
\item Graduated with distinction
\item Courses on stochastic processes, statistical mechanics, dynamical systems, cryptography, quantum computing, quantum mechanics, and scientific computation.
\end{itemize}

%\divider
\cvsection{Work}
\smallskip
\cvevent{Research Associate}{Cayuga Research}{May 2018 -- Present}{Waterloo, Canada}
\begin{itemize}
\item Worked as a part of a team for consulting work focused on the development and implementation of advanced optimization methods and data driven solutions to industrial problems. 
%\item Developed a global optimization toolbox in Matlab
\item Built prototype flight path optimization software able to plan flights that save up to 5\% in fuel costs compared to real commercial flights. 
\item Worked on a chiller plant optimization problem, developing data-driven models of the plant components and an optimization method that produced 4\% energy savings.
\end{itemize}

\divider

\cvevent{Summer Research Assistant}{Ayers Research Group, Department of Chemistry \& Chemical Biology, McMaster University}{ May 2015 -- Aug 2015}{Hamilton, Canada}
\begin{itemize}
\item Constructed equations constraining a two-electron reduced density matrix (2-RDM) to represent a many-electron quantum system.
\item Implemented a semi-definite optimization algorithm for constraining the density matrix.
\end{itemize}





%\end{fullwidth}
\end{document}
